% Exam Template for UMTYMP and Math Department courses
% http://mathcep.umn.edu/umtymp/
%
% Using Philip Hirschhorn's exam.cls: http://www-math.mit.edu/~psh/#ExamCls
%
% run pdflatex on a finished exam at least three times to do the grading table on front page.
%
%%%%%%%%%%%%%%%%%%%%%%%%%%%%%%%%%%%%%%%%%%%%%%%%%%%%%%%%%%%%%%%%%%%%%%%%%%%%%%%%%%%%%%%%%%%%%%

% These lines can probably stay unchanged, 
\documentclass[addpoints,11pt]{exam}
\RequirePackage{amssymb, amsfonts, amsmath, latexsym, verbatim, xspace, setspace}
% you can remove these two lines if you're not making pictures with tikz.
\RequirePackage{tikz}
\usetikzlibrary {plotmarks}

% By default LaTeX uses large margins.  This doesn't work well on exams; problems
% end up in the "middle" of the page, reducing the amount of space for students
% to work on them.
\usepackage[margin=1in]{geometry}


% Here's where you edit the Class, Exam, Date, etc.
\newcommand{\class}{Math 2B 44360}
\newcommand{\term}{Winter 2018}
\newcommand{\examnum}{Midterm 1}
\newcommand{\examdate}{Wed Jan 31 2018}
\newcommand{\examtime}{9.00am}

% For an exam, single spacing is most appropriate
\singlespacing
% \onehalfspacing
% \doublespacing

% For an exam, we generally want to turn off paragraph indentation
\parindent 0ex

\begin{document} 

% These commands set up the running header on the top of the exam pages
\pagestyle{head}
\firstpageheader{}{}{}
\runningheader{\class}{\examnum\ - Page \thepage\ of \numpages}{\examdate}
\runningheadrule

\begin{flushright}
\begin{tabular}{p{2.8in} r l}
\textbf{\class} & \textbf{Student's Name (Print):} & \makebox[1.5in]{\hrulefill}\\
\term & &\\
\examnum &\textbf{Student's ID:} & \makebox[1.5in]{\hrulefill}\\
\examdate &&\\
\examtime &  \textbf{Discussion Section Code:} & \makebox[1.5in]{\hrulefill}
\end{tabular}\\
\end{flushright}
\rule[1ex]{\textwidth}{.1pt}

{\bf Print your name and student ID on the top of this page.} \\

This exam contains \numpages\ pages (including this cover page) and
\numquestions\ problems. 
% {\bf Note that some equations are numbered.}  
You may \textit{not} use your books, notes, or any calculator in this exam. Do not write in the grading table below.\\

The following rules apply to the answers you provide in this exam:

\begin{minipage}[t]{3.7in}
\vspace{0pt}
\begin{itemize}

\item \textbf{Organize your work}, in a neat and coherent way.   

\item \textbf{Unsupported answers will not receive full credit}.  Calculation or verbal explanation is expected. 

\item \textbf{If you need more space, use the back of the pages}; clearly indicate when you have done this.

\item \textbf{Box your final answer} for full credit. 

\end{itemize}
\end{minipage}
\hfill
\begin{minipage}[t]{2.3in}
\vspace{0pt}
\gradetable[v]%[pages]  % Use [pages] to have grading table by page instead of question
\end{minipage}

\newpage % End of cover page

%%%%%%%%%%%%%%%%%%%%%%%%%%%%%%%%%%%%%%%%%%%%%%%%%%%%%%%%%%%%%%%%%%%%%%%%%%%%%%%%%%%%%
%
% See http://www-math.mit.edu/~psh/#ExamCls for full documentation, but the questions
% below give an idea of how to write questions [with parts] and have the points
% tracked automatically on the cover page.
%
%
%%%%%%%%%%%%%%%%%%%%%%%%%%%%%%%%%%%%%%%%%%%%%%%%%%%%%%%%%%%%%%%%%%%%%%%%%%%%%%%%%%%%%

\begin{questions} 

%%% L2
\question
\begin{parts} 
\part[5]
Estimate the area under the parabola $y=x^2$ from $x=0$ to $x=4$ using 4 approximating (Riemann) rectangles and right endpoints. 
\vfill 

\part[5]
Is this an upper bound or lower bound on the actual area? Illustrate why.
\vfill 

\part[5]
Using right endpoints, find an expression for the actual area as the limit of a Riemann sum. Do not evaluate your expression. 
\vfill 

\end{parts} 

\newpage 

\question
Evaluate the following: 
\begin{parts} 

%%% L5
\part[5]
\begin{eqnarray*}
\int (2 + \tan^2\theta) \, d\theta 
& \left[ \mbox{Hint: }
\frac{d}{d\theta} \tan\theta = \frac{1}{\cos^2\theta}.
\right]
\end{eqnarray*}
\vfill 

%%% L6 
\part[5]
\[ 
\int x^3 \sqrt{x^2 + 1} \, dx 
\]
\vfill 

%%% L6 
\part[5]
\[
\int \frac{\cos(\ln t)}{t} \, dt 
\] 
\vfill
\end{parts} 

\newpage 

%%% L6 
\question[5]
Given that $\int_0^9 f(x) \, dx = 4$, evaluate $\int_0^3 xf(x^2) \, dx$. 
\vfill 

%%% L4 
\question[5] 
Evaluate 
\[
\frac{d}{dx} \int_1^{e^x} \ln t \, dt .
\]
\vfill 


\question[5]
A particle moves along a line with velocity $v(t) = 3t - 5$ at time $t$. Find the displacement of the object in the time interval $[0, 3]$. 
\vfill 
\vfill

\newpage 

%%% L7
\question[5]
Compute the area of the region enclosed by $y=\sin x, y=x, x = \pi/2$ and $x = \pi$. 
\vfill 

%%% L8
\question 
\begin{parts} 
\part[5]
Find the volume of the solid obtained by rotating the region bounded by the curves 
$2x = y^2, x = 0$ and $y=4$ about the $y$-axis. 
\vfill 
\part[5]
Set up an integral to find the volume of the solid obtained by rotating the region bounded by $y = x^3, y=0$ and $x=1$ about the axis $x=2$. Do not evaluate the integral. 
\vfill 
\end{parts} 

\end{questions} 

\end{document}
